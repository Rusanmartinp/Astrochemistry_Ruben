\section{Introduction}

The study of binary systems, specially the close interacting ones is specially important to probe stellar physics, to understand how close binaries evolve and how a binary component affects its companion due to their short distance to the centre of mass. There is an important class of close interacting binaries called the Algol-type variables which consist on semi-detached binaries with intermediate mass components. In such systems the less massive star (hereafter donor) is more evolved than the most massive one (hereafter gainer) this paradox can be understood if the donor star used to be the massive star in the system, it evolved first and started a fast exchange of mass through Roche lobe overflow onto the gainer star as first studied by \citet{crawford1955} and later confirmed by numerical calculations by \citet{kippenhahn} and \citet{eggleton2006}.\\
\\
\indent This paper is focused in the double periodic variables (DPV) which are a sub-class of the Algol classification consisting in relatively massive stars with a gainer of $\sim 7 M_{\odot}$ and the donor star is filling its Roche lobe transfering mass to its companion. This type of binary were found in the Magellanic Clouds showing roughly sinusoidal periodic light variations with periods from 140 to 960 days \citep{mennickent2003}. They found a characteristic relation between long and short period given by $P_{long}= f \times P_{short}$ \citep{mennickent2003}. Here $f$ a correlation coefficient with an average value of $\sim 33$ but with single values for the period ratio tipically between $27$ and $39$ \citep{mennick2016a}. An interesting feature of this relation is that it is not only true for galactic DPVs but also for non-galactic ones. \citet{mennickent2003} found a $f=32.4$ for the original sample of the LMC and SMC DPVs. \citet{poleski2010} reported a value of $f=33.1$ for $125$ LMC DPVs. \citet{mennickent2012} reported a value of $f=32.7$ for thirteen galactic DPVs which were the only DPVs known at that time.
%-------------------------




%--------------------------
\indent In this paper, our goal is to further assess whether magnetic activity is a feasible alternative to explain the origin of such long period present in the DPVs system. In particular, we note that in the context of isolated stars, characteristic relations have been inferred between the activity cycle and the rotation period \citep{saar1999, bohm2007}, which can be interpreted in terms of simple dynamo models, as already proposed by \citet{soon1993} and \citet{baliunas1996}, they sugested a relation between rotation velocity, activity period and the dynamo number $D=\alpha \Delta\Omega d^{3}/\eta^{2}$. Here $d$ is the characteristic legth scale of the convection, $\eta$ is the turbulent magnetic diffusivity, $\Delta\Omega$ is the differential rotation and $\alpha$ is the magnetic helicity in the star. 